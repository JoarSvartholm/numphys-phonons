\documentclass[a4paper,11pt]{article}
\usepackage[utf8]{inputenc}            % Tekenkodning
\usepackage[T1]{fontenc}               % Fixa kopiering av texten
\usepackage[english]{babel}            % Språk (t.ex. Innehåll)
\usepackage{geometry}                  % Sidlayout m.m.
\usepackage{graphicx,epstopdf,float}   % Bilder
\usepackage{amsmath,amssymb,amsfonts}  % Matematik
\usepackage{enumerate}                 % Fler typer av listor
\usepackage{fancyhdr}                  % Sidhuvud/sidfot
\usepackage{hyperref}                  % Hyperlänkar
\usepackage{parskip}                   % noindent!!
\usepackage{float}                     % \begin{figure}[H] preciserar bildposition
\usepackage{subcaption}             % För att lägga figurer bredvidvarandra från: http://tex.stackexchange.com/questions/91224/placing-two-figures-side-by-side
\usepackage{textcomp}
\usepackage{gensymb}
\usepackage{wrapfig}
\usepackage[]{algorithm2e}
% packages for matlab codes
\usepackage{listings}
\usepackage{color}
\usepackage{pdflscape}
\usepackage{multicol}
\setlength{\columnsep}{0.6cm}
% Add new commands used-defined:
\newcommand{\m}[1]{\mathbf{#1}}
\newcommand{\tn}[1]{\textnormal{#1}}
\newcommand{\ve}[1]{\textnormal{vec}(#1)}


% instälningar för figurtexter
%\usepackage[margin=3ex,font=small,labelfont=bf,labelsep=endash]{caption}
\usepackage[font={small,it}]{caption}
\usepackage[labelfont={normal,bf}]{caption}
\usepackage[margin=3ex]{caption}

% mailadresser som hyperlänkar
\newcommand{\mail}[1]{\href{mailto:#1}{\nolinkurl{#1}}}
% Spara författare och titel
\let\oldAuthor\author
\renewcommand{\author}[1]{\newcommand{\myAuthor}{#1}\oldAuthor{#1}}
\let\oldTitle\title
\renewcommand{\title}[1]{\newcommand{\myTitle}{#1}\oldTitle{#1}}

% Hyperlänkar
\hypersetup{
  colorlinks   = true, %Colours links instead of ugly boxes
  urlcolor     = black, %Colour for external hyperlinks
  linkcolor    = black, %Colour of internal links
  citecolor   = black  %Colour of citations
}





\graphicspath{{./matlab/},{./images/},{./matlab/figs/}} % Söker också bilder i en undermapp figs.


%% DOCUMENT
%------------------------------------------------------------------%
\begin{document}
  \title{Phonons in Rare Gases}


  \author{
    Joar Svartholm - josv0150(\mail{josv0150@student.umu.se})\\
  }
  \date{\today}


\begin{titlepage}
  \maketitle
  \thispagestyle{fancy}
  \headheight 35pt
  \rhead{\small\today}
  \lhead{\small Department of Physics\\
    Umeå Universitet}



% State the aim of the experiment, what was measured, which techniques and methods were used, and the main result(s) and conclusion(s). Remember that the abstract should be understandable on its own, and you can thereby not refer to equations/figures/tables in the report. You should also not use references, since the information in the abstract should be available in the actual report.

  % Ändra till rätt namn m.m.
  \cfoot{Numerical Methods in Physics \\
  Supervisor: Claude Dion}

\end{titlepage}


\newpage
\pagestyle{fancy}
\headheight 30pt
\rhead{\small \myTitle\\\today}
\lhead{\small \myAuthor}
\cfoot{\thepage}

% Innehåll
\tableofcontents
\newpage


\section{The Dynamical Matrix}

The forst row in the dynamical matrix describing the motion of phonons in a rare gas can be derived as

\begin{equation}
  \label{eq:dynfirst}
\begin{aligned}
\omega^2\varepsilon_x = & \Big \{ \frac{1}{2m}[A+B][8-4\cos (k_x a)\cos (k_y a)-4\cos (k_x a)\cos (k_z a)] \\
 & + \frac{B}{m}[4-4\cos (k_y a)\cos (k_z a)] \Big \} \varepsilon_x \\
 & + \frac{1}{2m}[A-B]4\sin (k_x a)\sin (k_y a) \varepsilon_y \\
 & + \frac{1}{2m}[A-B]4\sin (k_x a)\sin (k_z a) \varepsilon_z
\end{aligned}
\end{equation}

By a cyclic permutation of this, using the fact that the coordinate axis are chosen arbritrarily the second and third row can be found as

\begin{equation*}
\begin{aligned}
 \omega^2\epsilon_y = & \frac{1}{2m}[A-B]4\sin (k_y a)\sin (k_x a) \varepsilon_x\\
&+ \Big \{ \frac{1}{2m}[A+B][8-4\cos (k_y a)\cos (k_z a)-4\cos (k_y a)\cos (k_x a)] \\
 & + \frac{B}{m}[4-4\cos (k_z a)\cos (k_x a)] \Big \} \epsilon_y \\
 & + \frac{1}{2m}[A-B]4\sin (k_y a)\sin (k_z a) \varepsilon_z \\
 \omega^2\epsilon_z =& \frac{1}{2m}[A-B]4\sin (k_z a)\sin (k_x a) \varepsilon_x \\
  & + \frac{1}{2m}[A-B]4\sin (k_z a)\sin (k_y a) \varepsilon_y\\
&+ \Big \{ \frac{1}{2m}[A+B][8-4\cos (k_z a)\cos (k_x a)-4\cos (k_z a)\cos (k_y a)] \\
 & + \frac{B}{m}[4-4\cos (k_x a)\cos (k_y a)] \Big \} \epsilon_z \\
\end{aligned}
\end{equation*}


\section{Dispersion relation}

\begin{figure}[H]
  \centering
  \includegraphics[width=0.8\textwidth]{figs/Kr_omega_100}
  \caption{Dispersion relation in the 100 direction}
  \label{fig:100}
\end{figure}

\begin{figure}[H]
  \centering
  \includegraphics[width=0.8\textwidth]{figs/Kr_omega_110}
  \caption{Dispersion relation in the 110 direction}
  \label{fig:110}
\end{figure}

\begin{figure}[H]
  \centering
  \includegraphics[width=0.8\textwidth]{figs/Kr_omega_111}
  \caption{Dispersion relation in the 111 direction}
  \label{fig:111}
\end{figure}



\section{End words}

All C code is found in the code folder;\newline \verb|/home/josv0150/Documents/numPhys/numPhys-FFT/code| including the code for plotting. If something needs to be recompiled simply type \verb| make gaussian| for the gaussian.c file etc. The plot functions are written in python3 and the line \verb|python3 plotDecoder.py| or equivalent should do the trick. I have not tested to run these codes on sesam since I uploaded the codes remotely and I have not tried it for other python versions either.



\end{document}
