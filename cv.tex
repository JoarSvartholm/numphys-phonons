\section{The heat capacity}

The heat capacity $C_V/V$ was calculated as a function of temperature for Ar, Ne, Kr and Xe and plotted which is shown in Fig. \ref{fig:heat}. As we can see in the figure the heat capacity reaches a high temperature limit. That is, a limit where the heat capacity is independent of temperature. This is predicted by the Dulong-Petit law

\begin{equation*}
  C_V = M k_B
\end{equation*}

where $M$ is the total degrees of freedom and $k_B$ is the boltzmann constant. The total degrees of freedom is for monatomic gases $3N$, where $N$ is the number of atoms. This yields

\begin{equation*}
  C_V / V = 3k_B \frac{N}{V} = 3k_B \frac{4}{V_c}
\end{equation*}

where $V_c$ is the volume of a unit cell and the factor 4 comes from the fact that there is 4 lattice points in a unit cell for a fcc-lattice. The analytic heat capacity is tabulated together with the limit value from the computed values in Tab. \ref{tab:heat}.

\begin{table}[h]
\centering
  \caption{Heat capacity in the high temperature limit.}
  \label{tab:heat}
  \begin{tabular}{l|rrr}
    & Analytic & Numeric & error[\%] \\ \hline
    Ne & 1863056 & 1860010 & 0.16\\
    Ar &  1112820 &  1109164 & 0.32\\
    Kr & 920624 &  918752 & 0.20\\
    Xe &  719237 & 718067 & 0.16
  \end{tabular}
\end{table}

\begin{figure}[H]
  \centering
  \includegraphics[width=0.8\textwidth]{figs/CV_highT}
  \caption{Heat capacity for different materials plotted against temperature.}
  \label{fig:heat}
\end{figure}
