\section{Dispersion relation}

The equations in the previous section yields a system of equations in which one can find the dispersion relation for phonons in a crystal. Solving this using

\begin{equation*}
  \m k = \frac{\pi c}{a}\m q
\end{equation*}

for several values of $c$ in the three axis of symmetry yields Fig. \ref{fig:100}-\ref{fig:111}. As one can see in the figures the curves are periodic which makes sense since the crystal is periodic with respect to a unit cell. Note that in Fig. \ref{fig:110} the curves appear to cross. Clearly the green curve should follow the orange path and vice versa after the crossing. This phonomena occur simply since the program puts the computed eigenfrequencies in decreasing order which makes it look like the curves bounces off eachother. In order to remove this one can simply find this point in the data series an swap the values between these two curves. Then one could easily follow one phonon mode. This in not done in this report since it is not of importance for this analysis.


\begin{figure}[H]
  \centering
  \includegraphics[width=0.8\textwidth]{figs/Kr_omega_100}
  \caption{Dispersion relation in the 100 direction}
  \label{fig:100}
\end{figure}

\begin{figure}[H]
  \centering
  \includegraphics[width=0.8\textwidth]{figs/Kr_omega_110}
  \caption{Dispersion relation in the 110 direction}
  \label{fig:110}
\end{figure}

\begin{figure}[H]
  \centering
  \includegraphics[width=0.8\textwidth]{figs/Kr_omega_111}
  \caption{Dispersion relation in the 111 direction}
  \label{fig:111}
\end{figure}

\section{The Gruneissen parameter}

The Gruneissen parameter

\begin{equation}
  \label{eq:Gruneissen}
  \gamma_j(\m q) = -\frac{\partial \ln (\omega (\m q,j))}{\partial \ln V}
\end{equation}

was calculated in the symmetri directions for Xenon to

\begin{equation*}
  \begin{split}
    [1 0 0] & \quad [3.02686 \; 3.02686 \; 3.381024] \\
    [1 1 0] & \quad [3.35004 \; 3.03490 \; 3.450501] \\
    [1 1 1] & \quad [2.677099 \; 2.677099 \; 3.470591]
  \end{split}
\end{equation*}
